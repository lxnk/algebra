% !TEX spellcheck = en-UK
% !TEX TS-program = pdflatex


\documentclass[11pt]{amsart}
\usepackage{amsthm}
\usepackage{fullpage}
\usepackage{amssymb}

\title{Interval arithmetics}
\author{Alex Kashuba}
%\date{}                                           % Activate to display a given date or no date

\DeclareMathOperator{\till}{.\!.}

\begin{document}

\maketitle

\section{Interval ring formulation}

\theoremstyle{definition}
\newtheorem*{defn}{Definition}
\theoremstyle{plain}
\newtheorem*{thm}{Theorem}
  
\begin{defn}[Formal interval]
A set $I$, whose elements are called formal intervals, together with a value map $v : \mathbb{I} \to \mathbb{IK}$ and a representation map $r: \mathbb{IK} \to \mathbb{I}$, and operations $\circ \in \{+,-,\times\}$ is called an interval ring (over $\mathbb{K}$) if the following axioms hold.
\end{defn}

\begin{thm}[Nonnegative intervals]
The set
$$
\mathbb{I} :=\{\langle a \rangle =\langle [\underline{x}\till\overline{x}],[\underline{y}\till\overline{y}] \rangle, \qquad 
\underline{x}\leqslant \overline{x}\leqslant 0 \leqslant\underline{y}\leqslant\overline{y}\}
$$
becomes an interval ring with the definitions
$$
v\langle a \rangle = [\underline{x}\till\overline{x}] + [\underline{y}\till\overline{y}] = 
[\underline{y}+\underline{x}\till \overline{y}+\overline{x}]
$$
Note that $\underline{y}+\underline{x} \leqslant \overline{y}+\overline{x}$
and
$$
r[\underline{x}\till\overline{x}] = 
\begin{cases}
\langle [0\till0],[\underline{x}\till\overline{x}] \rangle & \text{if $0 \leqslant\underline{x}\leqslant\overline{x}$} \\
\langle [\underline{x}\till\overline{x}],[0..0] \rangle & \text{if $\underline{x}\leqslant \overline{x}\leqslant 0$} \\
\langle [\underline{x}\till0],[0\till\overline{x}] \rangle & \text{if $\underline{x}\leqslant 0\leqslant \overline{x}$}
\end{cases}
$$
\end{thm}

\begin{align}
\langle [\underline{x}\till\overline{x}],[\underline{y}\till\overline{y}] \rangle + 
\langle [\underline{p}\till\overline{p}],[\underline{q}\till\overline{q}] \rangle =
\langle [\underline{x}+\underline{p}\till\overline{x}+\overline{p}],[\underline{y}+\underline{q}\till\overline{y}+\overline{q}] \rangle
\\
[\underline{y}+\underline{x}\till \overline{y}+\overline{x}] + 
[\underline{q}+\underline{p}\till\overline{q}+\overline{p}] =
[\underline{y}+\underline{q}+\underline{x}+\underline{p}\till\overline{y}+\overline{q} + \overline{x}+\overline{p}]
\end{align}

\begin{align}
- \langle [\underline{x}\till\overline{x}],[\underline{y}\till\overline{y}] \rangle =
\langle [-\overline{y}\till-\underline{y}], [-\overline{x}\till-\underline{x}] \rangle
\\
-[\underline{y}+\underline{x}\till \overline{y}+\overline{x}] = 
[-\overline{y}-\overline{x} \till -\underline{y}-\underline{x}] 
\end{align}


\begin{align}
\langle [\underline{x}\till\overline{x}],[\underline{y}\till\overline{y}] \rangle \times 
\langle [\underline{p}\till\overline{p}],[\underline{q}\till\overline{q}] \rangle =
\langle [\underline{x}\overline{q} + \overline{y}\underline{p} \till\overline{x}\underline{q} + \underline{y}\overline{p}],
[\underline{y}\underline{q}+\overline{x}\overline{p}\till\overline{y}\overline{q}+\underline{x}\underline{p}] \rangle
\\
[\underline{y}+\underline{x}\till \overline{y}+\overline{x}] \times 
[\underline{q}+\underline{p}\till\overline{q}+\overline{p}] =
[\min(A), \max(A)]
\\
A = \prod_{pair}  \{ (\underline{y}+\underline{x}),(\overline{y}+\overline{x});\,
(\underline{q}+\underline{p}),(\overline{q}+\overline{p}) \}
\end{align}


\end{document}  